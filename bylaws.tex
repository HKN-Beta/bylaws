\documentclass[10pt, oneside]{article}
\usepackage{geometry}
\usepackage{color,pxfonts,fix-cm}
\usepackage[T1]{fontenc}
\geometry{letterpaper}                   		% ... or a4paper or a5paper or ... 
\usepackage[parfill]{parskip}    		% Activate to begin paragraphs with an empty line rather than an indent
	
\usepackage{amssymb}
\usepackage{fancyhdr}

\usepackage{enumitem}
\setlist[enumerate]{nosep}
\setlist[itemize]{nosep}

\usepackage[hidelinks]{hyperref}

\usepackage{tocloft}
\renewcommand{\cfttoctitlefont}{\hfill\Large}
\renewcommand{\cftaftertoctitle}{\hfill}
\renewcommand{\cftsecnumwidth}{7em}
\renewcommand{\cftsubsecnumwidth}{6em}
\setcounter{tocdepth}{1}

\usepackage{titlesec}
\renewcommand{\thesection}{ARTICLE \Roman{section}}
\titleformat{\section}{\normalfont\Large\centering}{\thesection}{0.5em}{}
\renewcommand{\thesubsection}{SECTION \arabic{subsection}}
\titleformat{\subsection}[runin]{\normalfont\large\bfseries}{\thesubsection}{0.5em}{}
\renewcommand{\thesubsubsection}{\arabic{subsection}.\arabic{subsubsection}}
\titleformat{\subsubsection}[runin]{\normalfont\normalsize\itshape}{\thesubsubsection}{0.5em}{}


\title{Chapter Operations Manual \\ Beta Chapter \\ IEEE-Eta Kappa Nu Association \\~\\ Established: 1906}
\author{\\Revision 2.6}
\date{}

\begin{document}

\pagestyle{fancy}
\fancyhead[R]{}
\renewcommand{\sectionmark}[1]{\markright{\MakeUppercase{\thesection. \ #1}}{}}
\renewcommand{\subsectionmark}[1]{}


\maketitle
\thispagestyle{empty}
\clearpage
\setcounter{page}{1}

\tableofcontents
\clearpage

\section{Name, Charter Date, and Mission of the Chapter}
\subsection{}
The name of this Chapter is the Beta Chapter of IEEE-Eta Kappa Nu; hereafter referred to as the Chapter. On campus, the Chapter may be referred to as Beta Chapter of IEEE-Eta Kappa Nu; Beta Chapter, Eta Kappa Nu; HKN; IEEE-HKN; or IEEE-Eta Kappa Nu.
\subsection{}
This chapter shall be located at Purdue University, West Lafayette, IN.
\subsection{}
The Beta Chapter was chartered on October 28th 1904.
\subsection{}
IEEE-Eta Kappa Nu is the honor society of the Institute of Electrical and Electronics Engineers (IEEE). The mission of the Chapter is to recognize outstanding students, faculty, and professionals in IEEE technical fields of interest with a particular focus on the traditional fields of electrical engineering, electronic engineering, and computer science. IEEE-HKN chapters promote excellence in the profession and in education through an emphasis on scholarship, character, and attitude. Chapters further strive to foster community and cooperation among its active members and the host department(s), as well as other local honor societies and the local IEEE student branch.

\section{Governance of the Chapter}
\subsection{}
This \textbf{Chapter Operations Manual} shall be the official governing document of the Chapter and may be considered its constitution and/or bylaws. This Chapter Operations Manual is subordinate to the \textit{IEEE-Eta Kappa Nu Operations Manual} and the \textit{IEEE-Eta Kappa Nu Process Manual}. The most recent versions of these documents are available electronically from the IEEE-HKN website: \url{https://hkn.ieee.org}. IEEE-Eta Kappa Nu is an organizational unit of IEEE; IEEE policies, including the IEEE Code of Ethics, are available at: \url{http://ieee.org}.

\medskip

The rules governing this organization for conducting business shall be, in order of precedence:

\begin{itemize}
    \item The \textit{IEEE-Eta Kappa Nu Operations Manual}.
    \item The \textit{IEEE-Eta Kappa Nu Process Manual}.
    \item This \textbf{Chapter Operations Manual}.
    \item The latest version of \textit{Robert’s Rules of Order}.
\end{itemize}
\subsection{}
The Chapter members are divided into the following categories:

\begin{itemize}
    \item \textbf{Active members}: Current students who have been inducted into IEEE-Eta Kappa Nu, meet the eligibility requirements, and actively participate in Chapter activities. Active members are enrolled at the host university and have voting rights in Chapter decisions.

    \item \textbf{Alumni members}: Individuals who have graduated from the host university and were previously inducted as active Chapter members. Alumni members remain part of the IEEE-Eta Kappa Nu community and may attend Chapter events but do not have voting rights.

    \item \textbf{Advisors}: Faculty or industry professionals who support the Chapter by providing guidance and acting as liaisons between the Chapter, the host department, and IEEE-Eta Kappa Nu Headquarters. Advisors do not have voting rights but play a crucial role in mentoring and assisting Chapter members.

    \item \textbf{Officers}: Elected active members who hold specific positions within the Chapter, such as President, Vice-President, Treasurer, and Secretaries. Officers are responsible for the governance and daily operations of the Chapter, and they have voting rights on executive decisions.

    \item \textbf{Professionals}: Distinguished members of the community, such as faculty, industry leaders, or alumni who have made significant contributions to electrical and computer engineering fields. These members may be invited to participate in Chapter activities and events but do not have voting rights unless otherwise specified by Chapter regulations.
\end{itemize}
The voting members are ACTIVE MEMBERS, described in Section 3.


\subsection{}
For administrative and voting functions, the active Chapter members shall be those students and alumni who meet the following requirements (alumni are exempt from the enrollment requirement but must meet all other criteria):

\begin{itemize}
    \item Inducted into IEEE-Eta Kappa Nu.
    \item Enrolled in the host university for the \textbf{SEMESTER} in question and are not off campus for a cooperative assignment or industrial assignment.
    \item Have not missed two unexcused consecutive meetings
    \item Serve at least one weekly POD shift during the semester
    \item Complete an additional weekly POD shift or, with prior approval from the Executive Committee, fulfill a weekly hour of ECE-related or community service. Members of the Executive Committee are exempt from this requirement.

    \item Serve on at least one committee
    \item Complete four hours of volunteering service per semester
    \item Maintain an updated resume with the Chapter
\end{itemize}

\subsection{}
Amendments to this Chapter Operations Manual require a 2/3-majority vote of all active Chapter members, approval from any required university organizations, and approval by the IEEE-Eta Kappa Nu Board of Governors. Amendments may take effect immediately upon approval by the active membership of the Chapter. Amendments rejected by the IEEE-Eta Kappa Nu Board of Governors shall be returned to the Chapter and rendered void.

\section{Symbols and Media of the Chapter}
\subsection{}

The Chapter shall use the official symbols of IEEE-Eta Kappa Nu, including the emblem, shield, and colors as detailed in the \textit{IEEE-Eta Kappa Nu Operations Manual}.

\subsection{}

The \textbf{Beta} Chapter shall maintain its directory page at \url{https://hkn.ieee.org} and may also operate additional media platforms, including but not limited to a chapter website, university-sponsored web pages, and social media platforms such as LinkedIn and Instagram. These platforms shall be used to promote Chapter activities, engage members, and represent the Chapter publicly.

\section{Records, Faculty Advisor(s), and Trustees of the Chapter}
\subsection{}
The records of Beta Chapter, including chapter correspondence, membership signature book, financial records, checkbooks, chapter meeting minutes and other records deemed important, shall be located at Purdue University.
\subsection{}
The Chapter shall have at least one Faculty Advisor on the host department(s)’ faculty and may have additional Faculty Advisors or other Advisors such as industry Advisors.
Faculty Advisor(s) and other Advisors shall serve as liaisons between the Chapter and the university, as well as the Chapter and the industry Advisors (if any). The requirements and responsibilities of the Faculty Advisor(s) are detailed in the IEEE-Eta Kappa Nu Process Manual. All Advisors shall be approved by the head of the host department and voted on by active membership.

\subsection{}
A Trustees Committee shall be appointed by the head of the Chapter’s host department should the Chapter become inactive. The Trustees Committee shall be chaired by a member of the host department(s)’ faculty and the committee shall:
\begin{enumerate}[label=\alph*.]
\item Be custodian of all funds, records, and paraphernalia of the chapter.
\item Have full power, with the approval of the IEEE-Eta Kappa Nu Executive
Director, to reorganize and reactivate the chapter when it deems the time is appropriate.
\end{enumerate}
\subsection{}
Management of any chapter digital communications media shall only be performed by current chapter officers, or their duly appointed designee, and primary management permissions will be passed to the current facilities director at each election.

\section{Eligibility and Qualification}
\subsection{}
Undergraduate students, graduate students, and meritorious professionals such as faculty or distinguished alumni may be eligible for induction into IEEE-Eta Kappa Nu. The requirements for induction are detailed in the IEEE-Eta Kappa Nu Process Manual.
Undergraduate and graduate student candidates shall have been in regular attendance at Purdue University for at least one SEMESTER prior to the time of consideration.
% NOTE: The Credit/GPA requirement is in the Process Manual
\subsection{}
In addition to the requirements specified in the IEEE-HKN Process Manual, no candidate shall be considered for membership if they have on their record two failures in any of the electrical or computer engineering courses.

\subsection{}
Eligible students shall be invited to attend an orientation reception. The purpose of the reception is to acquaint the prospective candidates with IEEE-Eta Kappa Nu and requirements for election and induction, as well as to acquaint the current members with the prospective candidates.
\subsection{}
All prospective candidates shall exemplify strong character and a positive attitude as described in the IEEE-Eta Kappa Nu Induction Ritual and Process Manual. The Chapter may decline to induct academically eligible candidates if it determines, via a written policy approved by the IEEE-Eta Kappa Nu Board of Governors, that the candidate does not exemplify these values.

\subsection{}
Membership and participation are free from discrimination on the basis of race, religion, color, sex, age, national origin or ancestry, genetic information, marital status, parental status, sexual orientation, gender identity and expression, disability, status as a veteran, or any other legally protected status.

\section{Election, Requirements, and Induction}
\subsection{}
Candidates are individuals who have been nominated or identified for potential induction into the IEEE-Eta Kappa Nu Chapter. These individuals are typically students who satisfy specific academic and character criteria defined by the eligibility requirements of the Chapter. Candidates are not considered active members and do not have voting rights until they have fulfilled all requirements and have been formally inducted into the Chapter.

Students become candidates upon election by a majority vote of the active Chapter members. Considerations for this vote shall include, but shall not be limited to:

\begin{itemize}
    \item Compliance with the stated eligibility requirements related to academic record, character, and attitude.
    % \item Completion of the membership interview.
    \item Intent to complete candidate requirements and the induction.
\end{itemize}

Candidates shall be notified of their election within 48 hours after the election.

% \subsection{}
% Section 2: Each prospective candidate shall be interviewed before the election meeting to acquaint the active members with the candidate. Such interviews shall be held at a convenient time for both the candidate and the members. The interview shall be conducted in a dignified and serious manner, conforming with the ideals and aims of IEEE-Eta Kappa Nu.

\subsection{ }
Candidates shall complete the following requirements:
\begin{itemize}
    \item \textbf{Faculty Endorsements}: Obtain the signatures of at least four (4) faculty members of the Purdue University Electrical and Computer Engineering Department signifying their endorsement of the candidate.
    \item \textbf{Active Member Endorsements}: Obtain the signatures of at least six (6) active members of the Chapter.
    \begin{itemize}
        \item Two (2) of these signatures may be obtained through participation in and documentation of at least two (2) lounge duty sessions, as verified by the Operations Director or their designee.
    \end{itemize}
    \item \textbf{Advisor Meetings}: Schedule and attend a meeting with the Head Faculty Advisor(s) of the Chapter to discuss the responsibilities and expectations of membership.
    \item \textbf{Department Head Meeting}: Schedule and attend a meeting with the Head of the Purdue University Electrical and Computer Engineering Department (or their designated representative).
    \item \textbf{Executive Board Endorsements}: Obtain the signature of each current member of the Chapter's Executive Board, indicating their approval of the candidate.
    \item \textbf{Induction Fees}: Pay both the national Eta Kappa Nu (IEEE-HKN) induction fee and the local chapter induction fee as determined by the Executive Board.
    \item \textbf{Induction Ceremony Attendance}: Attend the official IEEE-HKN induction ceremony conducted by the Chapter, unless a formal excuse is approved by the Executive Board.
\end{itemize}
Candidates shall be notified regarding these requirements, particularly the schedule, procedure, and expectations for the induction.

\subsection{}

Faculty and professional candidates are distinguished individuals, such as faculty members, industry professionals, or alumni, who have demonstrated exceptional contributions to electrical and computer engineering or related IEEE technical fields.

\begin{enumerate}
    \item \textbf{Eligibility}: Faculty and professional candidates must exhibit outstanding character, leadership, and professional achievements that align with the values of IEEE-Eta Kappa Nu.

    \item \textbf{Nomination and Election}: Candidates in this category must be nominated by an active Chapter member or Advisor. The nomination must include a detailed description of the nominee’s achievements and contributions to the field. Faculty and professional candidates are elected through a majority vote of the active Chapter members.

    \item \textbf{Induction Ceremony}: Once elected, faculty and professional candidates must attend an induction ceremony. The ceremony shall follow the formal Induction Ritual as specified in the IEEE-Eta Kappa Nu guidelines. Membership becomes valid upon successful completion of the induction process and the submission of all necessary documentation and fees.

    \item \textbf{Membership Rights and Responsibilities}: Inducted faculty and professional members are granted all rights and privileges associated with membership in IEEE-Eta Kappa Nu, including participation in Chapter events and mentorship opportunities. However, they do not have voting rights in Chapter elections or business matters unless explicitly granted by the Chapter's governance policies.
\end{enumerate}

\subsection{}
MEMBERSHIP IN IEEE-ETA KAPPA NU IS ONLY VALID WHEN INDUCTION DOCUMENTATION AND FEES HAVE BEEN RECEIVED AND RECORDED BY
HEADQUARTERS’ STAFF AND ALL ACCURATE INDIVIDUAL INFORMATION HAS BEEN
ENTERED IN THE IEEE-ETA KAPPA NU DATABASE. Names and contact information for all candidates to be inducted shall be sent to IEEE-Eta Kappa Nu headquarters a minimum of three weeks prior to the induction date for membership certificates to be prepared for the induction ceremony.)

\subsection{}
The formal induction shall proceed according to the Induction Ritual shown on the IEEE-Eta Kappa Nu website. To be inducted into IEEE-Eta Kappa Nu, an inductee shall attend an induction ceremony.

\subsection{}
IEEE-Eta Kappa Nu headquarters shall be notified of inductees whose fees and information have been paid and recorded but who fail to attend the induction ceremony. These candidates may be inducted at a future ceremony within X month, and IEEE-HKN headquarters will re-issue the membership certificate with the new induction date.

\subsection{}
All who accept the invitation to membership shall be notified of the induction date, time, place, dress code, and procedures for the induction.


\section{Officers of the Chapter}
\subsection{} \label{Officers}
The Officers of the Chapter shall be the President, Vice-President, Treasurer, Recording Secretary, Public Relations Director, Events Director, Facilities Director, Operations Director, Recruitment Director, Volunteer Director, Lab Director, and President-Elect. [Other officer positions may be added by the Chapter.] The positions of Corresponding Secretary and News Correspondent shall be merged into the single office of Public Relations Director. Officer duties are detailed in the IEEE-Eta Kappa Nu Operations Manual and in the present article.

\subsection{}
Chapter officers shall be elected once every SEMESTER no later than the end of the SEMESTER. All newly elected officers shall take office at the conclusion of the last Chapter meeting for the SEMESTER, except for the President-Elect, who shall take office immediately following their election. The President-Elect will accede to the role of President at the conclusion of the last chapter meeting for the SEMESTER. 

\subsection{}
The outgoing Chapter officers shall provide IEEE-HKN Headquarters with the required submissions for their term, including induction documentation, fees, and election results. The election results report shall be submitted within one week of the election. The outgoing Chapter officers shall transfer officer materials and otherwise prepare incoming officers for their positions.

\subsection{}
Any vacancy among the officers for the Chapter shall be filled at the first regular meeting after the vacancy occurs or, when possible, before the vacancy takes effect.

\subsection{Eligibility for Officer Positions}\label{OfficerReq}
\subsubsection{General Eligibility}
Any nominee for an officer position within the Chapter must be a currently active member in good standing.

\subsubsection{Additional Eligibility for Specific Offices}
\begin{itemize}
    \item \textbf{Treasurer:} Nominees for the office of Treasurer must have satisfied all active member requirements for at least one (1) complete semester prior to the election.
    \item \textbf{President and Vice President:} Nominees for the offices of President or Vice President must have previously served a full term in at least one (1) other elected officer position within the Chapter.
\end{itemize}

\subsection{}
Nominations shall be conducted at the regular meeting preceding the regular meeting in which elections shall take place. Following the nominations meeting one week shall be allowed for further nominations. At the end of the one week interval the official ballot shall be made available to the Chapter (one week before the election meeting).

\subsection{Elections}

\subsubsection{Timing and Method}
Elections for all officer positions shall be held with sufficient time to ensure a smooth transition of responsibilities. The method of election shall be Instant Runoff Voting (IRV), conducted via a single ballot for all available officer positions. Each elected nominee may accept only one (1) officer position, even if elected to multiple. Elections shall take effect at the conclusion of the semester in which they are conducted.

\subsubsection{Presidential Election Process}
By the eighth week of the current semester, the current President shall announce their intention to either seek re-election for the following term or not.
\begin{itemize}
    \item \textbf{President Seeking Re-election:} If the President intends to run for re-election, the election for this position will occur at the same time and using the same Instant Runoff Voting method as all other officer positions.
    \item \textbf{President Not Seeking Re-election:} If the President does not intend to run for re-election, an election for the President-Elect position shall be held during the fifth (5th) Active Member Meeting of the current semester. The individual elected as President-Elect will then assume the role of President in the subsequent term.
\end{itemize}

\subsubsection{Subsequent Runoff Elections and Vacancies}
A subsequent runoff election will be held under the following circumstances:
\begin{enumerate}[label=\alph*.]
    \item If a candidate withdraws from the election, leaving a position without a winner.
    \item If there is a tie in the initial election results for a position, and none of the tied candidates are willing to withdraw after the results have been announced.
\end{enumerate}
Any subsequent runoff election shall ideally take place within the week following the initial election week, if not sooner.

\subsubsection{Filling Vacant Positions with No Candidates}
If an officer position has no candidates running in the general election, the Executive Committee shall nominate a candidate for that position. The nominated individual will then have the option to accept the nomination. The Executive Committee's nomination will be based on the nominee's previous engagement with the Chapter, considering factors such as Beta Beta Tau (Betabit) points and active/pledge awards. All eligibility requirements for the specific officer position (as outlined in \ref{OfficerReq}) must still be met by the nominated candidate.


\subsection{Officer Duties}

\subsubsection{President}
The duties of the President shall be to:
\begin{itemize}
    \item Provide overall leadership and strategic direction for the Chapter, ensuring its activities align with its mission and bylaws.
    \item Supervise the affairs of the Chapter, overseeing the work of all officers and committees to ensure effective operation.
    \item Preside over and maintain order (enforce decorum) at all meetings of the Chapter and the Executive Board, ensuring meetings are productive and follow established procedures.
    \item Serve as the primary representative and spokesperson for the Chapter when interfacing with other organizations, the university, and the public, fostering positive relationships.
    \item Ensure the timely completion and submission of the annual chapter report and any other required documentation.
    \item Attend the national student leadership conference during their term to represent the Chapter, network with other chapters, and gain valuable insights, or appoint a representative in their stead if unable to attend.
    \item Call special meetings of the Chapter or the Executive Board as deemed necessary or upon the request of a specified number of members as outlined in these bylaws.
    \item Work collaboratively with the other officers to develop and implement Chapter goals and initiatives.
    \item Act as a signatory for Chapter documents and financial transactions as authorized by the Executive Board.
    \item Perform other duties as necessary for the effective functioning and advancement of the Chapter.
\end{itemize}
\subsubsection{Vice President}
The duties of the Vice President shall be to:
\begin{itemize}
    \item Support the President in their duties and assume the responsibilities of the President in their absence or inability to serve.
    \item Oversee the chairs of any other committees that do not report to another specific officer, ensuring effective communication and collaboration.
    \item Attend the national student leadership conference during their term to represent the Chapter and gather valuable insights, or appoint a representative in their stead if unable to attend.
    \item Assist in the development and implementation of the Chapter's strategic goals and initiatives.
    \item Perform other duties as assigned by the President or the Executive Board.
\end{itemize}
\subsubsection{Recording Secretary}
The duties of the Recording Secretary shall be to:
\begin{itemize}
    \item Maintain records of all meetings of the Chapter.
    \item Ensure timely distribution of meeting minutes to members.
    \item Provide adequate notice for all special meetings of the Chapter.
    \item Keep the Chapter's schedule current and accessible.
    \item Maintain an up-to-date list of active members.
\end{itemize}
\subsubsection{Public Relations Director}
The duties of the Public Relations Director shall be to:
\begin{itemize}
    \item Develop and implement the Chapter's social media strategy to engage members, promote events, and enhance the Chapter's online presence.
    \item Coordinate and execute promotional activities to raise awareness of the Chapter and its initiatives among students and the wider community.
    \item Serve as a representative of the Chapter at internal university events and with student organizations to build connections and promote collaboration.
    \item Coordinate the creation and dissemination of publicity materials, including flyers, posters, and digital content.
    \item Manage the Chapter's contributions to THE BRIDGE and maintain the Chapter's presence on the IEEE-HKN website, ensuring information is current and engaging.
    \item Prepare and submit reports of the Chapter’s activities to the IEEE-HKN Board of Governors (BOG) and ensure compliance with reporting requirements.
    \item Oversee the Public Relations Committee chair, providing guidance and ensuring the timely publication of informative and engaging newsletters for the Chapter's members and stakeholders.
    \item Coordinate for general correspondence of the Chapter related to promotional activities and external engagement, excluding financial and event-specific communications.
\end{itemize}
\subsubsection{Treasurer}
The duties of the Treasurer shall be to:
\begin{itemize}
    \item Collect all membership dues and other income owed to the Chapter, ensuring accurate record-keeping of payments received.
    \item Maintain accurate and up-to-date financial accounts for the Chapter, documenting all income and expenditures in accordance with the guidelines and procedures established by the Business Office for Student Organizations (BOSO) at Purdue University.
    \item Deposit all of the organization's funds into the bank account(s) officially recognized by BOSO, ensuring timely and secure handling of finances.
    \item Make expenditures on behalf of the Chapter only in a manner that is properly authorized by the Executive Board and adheres strictly to the regulations and approval processes set forth by BOSO.
    \item Develop and present a proposed annual budget to the Executive Board for review and approval, and provide regular updates on the Chapter's financial status.
    \item Enforce the approved budget, ensuring that all expenditures are within the allocated limits and in line with the Chapter's financial policies.
    \item Prepare regular financial reports (e.g., monthly, semesterly) for presentation at meetings of the Executive Board and the general membership, detailing income, expenses, and the overall financial health of the Chapter.
    \item Maintain all necessary financial documentation, including receipts, invoices, and bank statements, in an organized and accessible manner, adhering to BOSO record-keeping requirements.
    \item Serve as the primary point of contact for the Chapter regarding financial matters with BOSO and other relevant university offices.
    \item Ensure compliance with all financial policies and procedures of the University and BOSO.
\end{itemize}
\subsubsection{Events Director}
The duties of the Events Director shall be to:
\begin{itemize}
    \item Serve as the primary point of contact and submit all necessary correspondence with the Purdue University Office of Student Activities and Organizations (SAO) regarding event planning and approvals.
    \item Oversee the planning, coordination, and execution of all social, professional development, and other events for the Chapter.
    \item Ensure that all planned events comply with University regulations, SAO policies, and Chapter bylaws.
    \item Work closely with the Public Relations Director to develop and implement effective advertising and promotional strategies for all Chapter events.
    \item Evaluate the success of each event and provide recommendations for future improvements.
    \item Keep the Executive Board informed of the progress and outcomes of event planning.
    \item Perform other duties related to Chapter events as needed.
\end{itemize}
\subsubsection{Facilities Director}
The duties of the Facilities Director shall be to:
\begin{itemize}
    \item Serve as the primary liaison with the Electrical Engineering Building Deputy and ECN on all matters related to the Chapter's physical spaces and infrastructure.
    \item Oversee the strategic planning and development of the Chapter's physical and technological resources, ensuring they meet the needs of the membership.
    \item Ensure compliance with all relevant building regulations and safety protocols within the Chapter's spaces.
    \item Keep the Executive Board informed of the status of facilities and IT infrastructure.
    \item Perform other duties related to the Chapter's physical and technological resources as needed.
\end{itemize}
\subsubsection{Operations Director}
The duties of the Operations Director shall be to:
\begin{itemize}
    \item Oversee the development and maintenance of the lounge's operating schedule, ensuring adequate coverage and accessibility for members.
    \item Oversee the procurement, restocking, and inventory management of lounge supplies, ensuring the lounge is well-equipped for member use.
    \item Coordinate with the Treasurer on budget matters related to lounge operations and supplies.
    \item Ensure the lounge environment is welcoming, organized, and adheres to any relevant Chapter policies.
    \item Regularly assess the operational needs of the lounge and propose improvements to the Executive Board.
    \item Keep the Executive Board informed of the operational status of the lounge.
    \item Perform other duties related to the operation of the Chapter's lounge as needed.
\end{itemize}
\subsubsection{Recruitment Director}
The duties of the Recruitment Director shall be to:
\begin{itemize}
    \item Develop and implement comprehensive recruitment strategies to attract eligible students to the Chapter.
    \item Coordinate and oversee all recruitment events and activities, such as information sessions, outreach programs, and collaborations with other student organizations.
    \item Serve as the primary point of contact for prospective members, providing information and answering inquiries about the Chapter and the induction process.
    \item Manage the application and selection process for new members, ensuring fairness and adherence to Chapter bylaws and Eta Kappa Nu (HKN) requirements.
    \item Collect and submit all required new member information and fees to Eta Kappa Nu Nationals in a timely and accurate manner.
    \item Work closely with the Public Relations Director to promote recruitment events and the benefits of membership through various communication channels.
    \item Maintain records of prospective members and recruitment efforts to track effectiveness and identify areas for improvement.
    \item Regularly report on recruitment activities and progress to the Executive Board.
    \item Perform other duties related to membership recruitment and onboarding as needed.
\end{itemize}
\subsubsection{Volunteer Director}
The duties of the Volunteer Director shall be to:
\begin{itemize}
    \item Develop and promote a comprehensive vision for the Chapter's volunteer and service activities, aligning with the values of Eta Kappa Nu and the needs of the community and the Electrical and Computer Engineering (ECE) department.
    \item Identify, evaluate, and propose potential volunteer opportunities that align with the Chapter's mission and the interests of its members.
    \item Coordinate logistics for volunteer events, including scheduling, transportation, and necessary resources.
    \item Work collaboratively with the Public Relations Director to publicize volunteer opportunities and celebrate the Chapter's service contributions.
    \item Maintain records of member participation in volunteer activities and the impact of service projects.
    \item Foster partnerships with local community organizations and the ECE department to identify and support service needs.
    \item Regularly report on the activities and outcomes of the Volunteering Committee to the Executive Board.
    \item Encourage and motivate Chapter members to actively participate in volunteer and service initiatives.
    \item Perform other duties related to the Chapter's volunteer efforts as needed.
\end{itemize}
\subsubsection{Lab Director}
The duties of the Lab Director shall be to:
\begin{itemize}
    \item Serve as the primary point of contact and liaison with other clubs or organizations that share the lab space, fostering positive relationships and coordinating shared resources.
    \item Attend all required safety meetings related to the lab space and ensure that the Chapter's use of the lab adheres to all safety protocols and regulations.
    \item Oversee the maintenance, organization, and cleanliness of the Chapter's lab space, ensuring a safe and productive environment for members.
    \item Coordinate with the Facilities Director on any issues related to the physical infrastructure and equipment within the lab space.
    \item Manage access to the lab space for authorized members, potentially including scheduling and key management, in accordance with Chapter policies.
    \item Work with the Treasurer on budget matters related to lab supplies, equipment maintenance, and upgrades.
    \item Ensure that all members using the lab space are aware of and adhere to safety guidelines and operating procedures.
    \item Regularly report on the status of the lab space, safety concerns, and interactions with other user groups to the Executive Board.
    \item Perform other duties related to the Chapter's lab space as needed.
\end{itemize}
\subsubsection{President-Elect}
The duties of the President-Elect shall be to:
\begin{itemize}
    \item Actively observe and learn from the President in the execution of all their duties, including attending all Chapter and Executive Board meetings, reviewing communications, and participating in planning sessions where appropriate.
    \item Work closely with the President to gain a comprehensive understanding of the responsibilities, procedures, and ongoing projects of the President's role before formally transitioning into the position.
    \item Assist the President with specific tasks or projects as assigned to gain practical experience and contribute to the Chapter's operations.
    \item Familiarize themselves with the Chapter's bylaws, policies, and historical records to build a strong foundation of knowledge.
    \item Begin to develop their vision and goals for their upcoming term as President, potentially including strategic initiatives and committee priorities.
    \item Participate in leadership training or development opportunities as available and relevant.
     \item Maintain all responsibilities and actively fulfill the duties of their current elected or appointed position on the Executive Board.
    \item Represent the Chapter or the President at specific events or meetings if requested and prepared to do so.
    \item Prepare for the formal transition of power, including meeting with outgoing officers and reviewing relevant documentation.
    \item Perform other duties as assigned by the President or the Executive Board to aid in their training and the Chapter's functioning.
\end{itemize}

\section{Executive Committee}
\subsection{}
Voting members of the Executive Committee shall consist of the officers mentioned in \ref{Officers}. The Faculty Advisor(s) and other Advisors shall be non-voting members.

\subsection{}
The Executive Committee shall have the power to make decisions affecting the day-to-day operation of the Chapter between regularly scheduled Chapter meetings.

\subsection{}
The Executive Committee shall serve as a planning committee for activities throughout the semester and shall propose the initial agendas for the regular or special Chapter meetings.

\subsection{}
A quorum for the transaction of business at an Executive Committee meeting shall be 50\%.

\subsection{}
The Faculty Advisor(s) and other Advisors shall be invited to all Executive Committee meetings, but are not required to attend.

\subsection{}
Decisions of the Executive Committee may be overturned by a majority vote of the active Chapter members.

\subsection{}
The Executive Committee shall be authorized to make expenditures of \$200 or less for goods and services necessary for the operation of the Chapter without a vote of the entire membership.

\subsection{}
The executive committee shall meet before each regular and special meeting of the chapter, and additionally at least one semester planning meeting before the end of the first week of the semester.

\subsection{}
The President of the chapter shall preside as the chair of the executive committee.



\section{Committees of the Chapter}
\subsection{}
The standing committees of the Chapter are mentioned in \ref{committees} . In addition to these standing committees, ad hoc committees may be created:
\begin{itemize}
    \item By vote of the Executive Committee
    \item OR By a vote of the Chapter active members
\end{itemize}


\subsection{}
Each executive officer will be appointed the chair of their respective committee. Other committee chairs shall:
\begin{itemize}
    \item Be elected by the Chapter members
    \item OR Be appointed by the Chapter President
    \item OR Be elected by committee members
\end{itemize}

Committee membership shall:
\begin{itemize}
    \item Be on a volunteer basis by each Chapter member
    \item OR Be appointed by the Chapter President.
\end{itemize}

\subsection{ } \label{committees}
The Standing Committees shall be:
\begin{enumerate}[label=\alph*.]
    \item Clerical
    \item Events
    \item Facilities
    \item Lab
    \item Operations
    \item Public Relations
    \item Recruitment
    \item Treasury
    \item VP
    \item Volunteering
\end{enumerate}

\subsection{Duties}
The duties of the Standing Committees shall be defined as follows.

\subsubsection{Clerical Committee}
The Clerical Committee shall operate under the direction of the Recording Secretary and shall be responsible for:
\begin{itemize}
    \item Chronicle the Chapter's activities and maintain a historical record of significant events, projects, and achievements.
    \item Organize and preserve the Chapter's archives, including documents, photographs, and other important materials, ensuring their accessibility.
    \item Assist the Recording Secretary in maintaining accurate membership records.
    \item Support the Recording Secretary in the distribution of meeting minutes and other official communications.
    \item Aid in the upkeep of the Chapter's calendar and scheduling information.
    \item Assist the Recording Secretary in their duties related to record-keeping and communication.
    \item Perform other clerical tasks as assigned by the Recording Secretary or the Executive Board.
\end{itemize}

\subsubsection{Events Committee}
The Events Committee shall operate under the direction of the Events Director and shall be responsible for:
\begin{itemize}
    \item Plan and coordinate a diverse range of social activities for the Chapter membership to foster camaraderie and engagement.
    \item Organize and execute the semesterly picnic or a similar large social gathering for the Chapter.
    \item Develop a semesterly calendar of events that includes a minimum of twelve (12) activities for the Chapter, encompassing social, professional development, and other relevant opportunities.
    \item Assist the Events Director in all logistical aspects of event planning and execution, including venue booking, catering arrangements, and setup.
    \item Brainstorm and propose ideas for new and engaging events for the Chapter.
    \item Support the Events Director in communicating event details and progress to the Executive Board.
    \item Assist the Events Director in other duties related to Chapter events as needed.
\end{itemize}

\subsubsection{Facilities Committee}
The Facilities Committee shall operate under the direction of the Facilities Director and shall be responsible for:
\begin{itemize}
    \item Maintain the physical information infrastructure necessary for the Chapter's operations, including network connectivity and hardware.
    \item Maintain the Chapter's website infrastructure, ensuring its functionality, security, and upkeep.
    \item Develop and maintain a standard operating procedure (SOP) wiki or similar knowledge base for Chapter processes and technical resources.
    \item Maintain the information technology infrastructure required for the operation of the Chapter's lounge, such as any dedicated systems or software.
    \item Maintain the physical infrastructure of the Chapter's lounge, including furniture, fixtures, and overall upkeep.
    \item Maintain the physical infrastructure and equipment of the Chapter-sponsored lab space, ensuring its functionality and safety.
    \item Assist the Facilities Director in strategic planning and development projects related to the Chapter's physical and technological resources.
    \item Support the Facilities Director in ensuring compliance with building regulations and safety protocols within the Chapter's spaces.
    \item Report any maintenance needs or infrastructure issues to the Facilities Director in a timely manner.
    \item Assist the Facilities Director in other duties related to the Chapter's physical and technological resources as needed.
\end{itemize}

\subsubsection{Lab Committee}
The Lab Committee shall operate under the direction of the Lab Director and shall be responsible for:
\begin{itemize}
    \item Maintain the organization, cleanliness, and functionality of the student lab space sponsored by the Chapter, ensuring a productive and safe environment for all users.
    \item Assist the Lab Director in overseeing the maintenance of equipment within the lab space, reporting any malfunctions or repair needs promptly.
    \item Help enforce safety guidelines and operating procedures within the lab space, promoting a culture of safety and responsibility among users.
    \item Assist the Lab Director in liaising with other clubs or organizations that share the lab space, facilitating communication and coordination of shared resources.
    \item Support the Lab Director in conducting regular inventories of lab supplies and equipment.
    \item Plan and conduct a minimum of two (2) workshops or training sessions per semester within the lab space to enhance members' technical skills and knowledge.
    \item Provide input to the Lab Director on potential improvements or upgrades to the lab space and its equipment.
    \item Assist the Lab Director in other duties related to the Chapter's lab space as needed.
\end{itemize}

\subsubsection{Operations Committee}
The Operations Committee shall operate under the direction of the Operations Director and shall be responsible for:
\begin{itemize}
    \item Manage the ordering and restocking of lounge inventory, including snacks, beverages, and other necessary supplies, in accordance with budgetary guidelines.
    \item Conduct inventory checks of lounge supplies at least every two weeks during the academic semester to monitor stock levels and identify restocking needs.
    \item Assist the Operations Director in developing and maintaining the lounge's operating schedule, ensuring adequate coverage and accessibility for members.
    \item Help maintain the organization and cleanliness of the lounge space, contributing to a positive environment.
    \item Assist the Operations Director in assessing the operational needs of the lounge and proposing suggestions for improvements to the Executive Board.
    \item Help ensure that all lounge operations adhere to relevant Chapter policies and guidelines.
    \item Communicate any operational issues or supply needs to the Operations Director in a timely manner.
    \item Assist the Operations Director in other duties related to the operation of the Chapter's lounge as needed.
\end{itemize}

\subsubsection{Public Relations Committee}
The Public Relations Committee shall operate under the direction of the Public Relations Director and shall be responsible for:
\begin{itemize}
    \item Publish a monthly newsletter to keep Chapter members informed of activities, opportunities, and important announcements.
    \item Publish a comprehensive Chapter newsletter each semester to highlight achievements, upcoming events, and member spotlights for a broader audience.
    \item Prepare and submit articles about the Chapter's activities and successes to THE BRIDGE, the official publication of IEEE-HKN, during each semester.
    \item Assist the Public Relations Director in developing and implementing the Chapter's social media strategy, creating engaging content and managing the Chapter's online presence.
    \item Support the Public Relations Director in coordinating and executing promotional activities for the Chapter and its events, both online and offline.
    \item Assist in the creation of publicity materials, such as flyers, posters, and digital content, to promote the Chapter's brand and activities.
    \item Help maintain and update the Chapter's website to ensure information is current, accurate, and engaging for visitors.
    \item Assist the Public Relations Director in preparing reports of the Chapter's activities for submission to the IEEE-HKN Board of Governors (BOG).
    \item Perform other public relations related tasks as assigned by the Public Relations Director or the Executive Board.
\end{itemize}

\subsubsection{Recruitment Committee}
The Recruitmemt Committee shall operate under the direction of the Recruitment Director and shall be responsible for:
\begin{itemize}
    \item Coordinate and conduct interviews of prospective initiate candidates to assess their qualifications and fit for membership.
    \item Oversee and track each stage of the Chapter's initiation process, ensuring all requirements are met by the candidates in a timely and organized manner.
    \item Coordinate the planning and execution of the induction ceremony for new members, ensuring a meaningful and celebratory event.
    \item Assist the Recruitment Director in managing the application and selection process for new members.
    \item Maintain clear communication with prospective initiates throughout the entire process, providing necessary information and guidance.
    \item Assist the Recruitment Director in collecting and submitting required new member information and fees to Eta Kappa Nu Nationals.
    \item Collaborate with the Public Relations Director to promote the induction ceremony and celebrate the new members.
    \item Assist the Recruitment Director in other duties related to membership recruitment and onboarding as needed.
\end{itemize}

\subsubsection{Treasury Committee}
The Treasury Committee shall operate under the direction of the Treasurer and shall be responsible for:
\begin{itemize}
    \item Manage the cycling of the lounge cash fund on a daily basis, ensuring accurate handling of transactions and reconciliation of the fund.
    \item Assist the Treasurer in the overall management of the Chapter's funds and accounts, including tracking income and expenditures.
    \item Support the Treasurer in collecting membership dues and other income owed to the Chapter, potentially assisting with record-keeping and follow-up.
    \item Help the Treasurer in the development of the annual budget by gathering financial data and projections.
    \item Assist in organizing and maintaining the Chapter's financial documentation, such as receipts and invoices.
    \item Serve as a point of contact for members with routine inquiries related to financial matters, as directed by the Treasurer.
    \item Assist the Treasurer in other financial tasks as needed.
\end{itemize}

\subsubsection{VP Committee}
The VP Committee shall operate under the direction of the Vice President and shall be responsible for:
\begin{itemize}
      \item Planning and coordinating a semesterly banquet for the Chapter to celebrate achievements and foster community among members.
    \item Conducting the end-of-semester audit of the Chapter's financial accounts to ensure accuracy and transparency, and reporting findings to the Vice President and the Executive Board.
    \item Recruiting and engaging Chapter members to actively participate in committee activities and events.
    \item Regularly reporting on the progress and outcomes of committee activities to the Vice President.
    \item Performing other duties as assigned by the Vice President or the Executive Board.
\end{itemize}

\subsubsection{Volunteering Committee}
The Volunteering Committee shall operate under the direction of the Volunteer Director and shall be responsible for:
\begin{itemize}
    \item Plan and coordinate all Electrical and Computer Engineering (ECE) outreach events initiated by the Chapter to engage prospective students or the broader community.
    \item Schedule and execute a minimum of one (1) educational outreach event per semester.
    \item Plan and coordinate all ECE service activities for the benefit of current ECE students, addressing their academic, professional, or community needs.
    \item Schedule and execute a minimum of eight (8) service activities per semester aimed at benefiting ECE students.
    \item Assist the Volunteer Director in identifying, evaluating, and proposing potential volunteer opportunities that align with the Chapter's mission and member interests, both within the ECE department and the broader community.
    \item Support the Volunteer Director in coordinating the logistics for volunteer events, including scheduling, transportation, and resources.
    \item Work with the Public Relations Director to promote volunteer opportunities and highlight the Chapter's service contributions.
    \item Assist the Volunteer Director in maintaining records of member participation in volunteer activities and the impact of service projects.
    \item Assist the Volunteer Director in other duties related to the Chapter's volunteer efforts as needed.
\end{itemize}

\section{Meetings of the Chapter}
\subsection{}
A quorum for the transaction of Chapter business shall consist of at least 50\% of the active Chapter members. Student members pursuing a cooperative program or those who are off the campus on an industrial assignment at the time of a meeting shall not be counted in the total membership for the purpose of determining a quorum. In addition, active members with approved, full-semester conflicts with chapter meetings, as acknowledged by the executive committee, are likewise not counted in the total membership for the purpose of determining quorum.

\subsection{}
A meeting schedule shall be published at the beginning of each SEMESTER. A minimum of eight regular meetings shall be held each semester.

\subsection{}
The Chapter President may call a special meeting at any time and shall be required to call a special meeting within two weeks upon request of five active members or the Faculty Advisor(s).

\subsection{}
Except as provided in this Operations Manual, all questions of order shall be decided by the Executive Committee.

\subsection{}
The order of a general business meeting shall be as follows:
\begin{enumerate}[label=\alph*.]
\item Roll Call
\item Reading of the previous meeting’s minutes
\item Officer Reports
\item Committee Reports
\item Old Business
\item New Business
\item Election of Officers (if on agenda)
\item Election of New members (if on agenda)
\item Appointment of Committees (if needed)
\item Special Papers and Presentations
\item Announcements, Discussion
\end{enumerate}

\subsection{}
All absences from meetings by members of the Chapter shall be unexcused unless approved by the Executive Committee. All excused absences shall be presented at the Chapter meeting with the reason for the absence.

\section{Dues, Fees, and Assessments}
\subsection{}
The induction fee shall be determined by the IEEE-Eta Kappa Nu Board of Governors and published by IEEE-Eta Kappa Nu Headquarters.

\subsection{}
A local induction fee may be accessed by the Chapter in addition to the induction fee. This fee shall be set by a majority vote of the Chapter officers and may not increase by more than 10\% between two adjacent academic terms. The fee set by the Chapter officers shall be reported to the Chapter at the next Chapter meeting. A simple majority of the Chapter shall either confirm or modify the value of the local induction fee. If a new fee is not proposed, the previous fee will be assumed.

\subsection{}
The Chapter‘s Executive Committee, at their discretion, may waive an individual’s fees with just cause. The Treasurer shall be instructed to pay the induction fee from Chapter funds.

\section{Funds of the Chapter}
\subsection{}
The Chapter shall maintain a general fund to pay all operating expenses of the Chapter. The general fund shall be used for all Chapter dues, fees, assessments, bank interest, and proceeds collected from other Chapter activities. The general fund shall be used to pay all Chapter operating expenses.

\subsection{}
Expenditures from the general fund and associated accounts of the Chapter shall be approved by a majority vote of the active Chapter members in the form of a semester or annual budget or may be approved by the Executive Committee if under \$200.

\subsection{}
Accounts for the Chapter shall be held with the BOSO at Purdue University.
Multiple accounts may be held to separate funds for various Chapter programs and the accounts shall be interest-earning whenever possible. Creation or changes to Chapter accounts shall be approved by the active membership of the Chapter and the university.

\subsection{}
The name on all Chapter accounts shall be “Beta Chapter of IEEE-Eta Kappa Nu” and the authorized signers on the account(s) shall be the Treasurer, the President, and at least one of the Faculty Advisor(s) or other Advisors. All withdrawals and disbursements shall be approved by at least two authorized signers.
\subsection{}
The Treasurer shall manage the Chapter’s account(s) and shall maintain financial records in accordance with the policies outlined in the IEEE-HKN Process Manual as well as any regulations specified by University. The Treasurer shall prepare a financial report for each regular meeting of the Chapter. The financial report shall include current balances, reports of deposits and expenditures since the last meeting, and an estimate of upcoming deposits and expenditures. The treasurer shall keep the books open to inspection by any active Chapter member.

\subsection{}
The newly-elected Treasurer and one other member appointed by the Faculty Advisor(s) shall perform an audit of the financial records at the end of the Treasurer’s term.

\subsection{}
The fiscal year for reporting revenue and expenses shall be July 1 through June 30.

\subsection{}
The Treasurer shall be responsible for filing the appropriate tax forms and reports as specified in the IEEE-HKN Process Manual.

\subsection{}
Monies shall be deposited in an account approved by the university and the officers of the Chapter.

\subsection{}
For bank accounts that require a Tax ID number, the Tax ID number for Beta Chapter shall be used. This Tax ID number shall be determined by the Chapter and the Purdue University Business Office for Student Organizations. Chapter accounts shall not be established in any member’s name, nor should the social security number of any member or faculty adviser be used on chapter accounts.

\subsection{}
A Budget shall be required for each semester. Funds allocated via the budget shall not require further approval by the chapter upon ratification of the budget. The ratification of the budget is to be through a majority vote of the chapter.

\section{Officer Transitions}
\subsection{}
To ensure a smooth transition between outgoing and incoming officers, the Chapter shall implement a structured transition process. This process is designed to maintain continuity in leadership, preserve institutional knowledge, and prepare newly elected officers for their responsibilities.

\subsection{}
The officer transition process shall begin immediately following the election of new officers and shall be completed before the end of the current academic term. The outgoing officers shall remain available to support the incoming officers throughout this period.

\subsection{}
Outgoing and incoming officers shall hold at least one joint transition meeting to discuss the following topics:
\begin{itemize}
    \item Responsibilities and duties of each officer position.
    \item Ongoing and upcoming projects or initiatives.
    \item Important contacts, resources, and Chapter documents.
    \item Budgetary and financial information, including account balances and expenditure plans.
    \item Challenges and successes from the previous term and recommendations for the future.
\end{itemize}

\subsection{}
Outgoing officers shall provide incoming officers with all relevant documentation and resources, including:
\begin{itemize}
    \item Meeting minutes, reports, and correspondence.
    \item Financial records and budget plans.
    \item Chapter operations manual and other governing documents.
    \item Access credentials for Chapter accounts and online platforms.
\end{itemize}



\section{Honors}
\subsection{}
Upon graduation, having completed at least one semester of active membership in the Chapter, active members are entitled to receive a paddle and a stole. The cost of the paddle and stole shall be covered for one graduation. Upon additional graduations, graduates are eligible to purchase a paddle and a stole from the Chapter at their own expense.

\subsection{}
Upon the satisfactory completion of a full term of office, each Chapter officer shall be entitled to receive a lapel pin recognizing their service. Additionally, the outgoing President shall receive a gavel or a pen, and the outgoing Vice President shall receive a pen as further acknowledgement of their roles.

\subsection{}
All chapter Honors unless otherwise stated in the Bylaws shall be presented in the form of a plaque or a medal.

\clearpage

{\normalfont\Large \centering Approvals: \\~\\}

\noindent The above Chapter Operations Manual was approved by the members of the Beta
Chapter at a [regular/special] business meeting held on \rule{4cm}{0.4pt}, with \rule{1cm}{0.4pt} members present and was approved by a vote of \rule{1cm}{0.4pt} in favor and \rule{1cm}{0.4pt} opposed. This Operations Manual shall become effective on \rule{4cm}{0.4pt}.
\\[2\baselineskip]

\noindent \rule{8cm}{0.4pt} \hfill {\rule{4cm}{0.4pt} \\
\noindent Chapter President \hfill Date
\\[2\baselineskip]
\noindent \rule{8cm}{0.4pt} \hfill {\rule{4cm}{0.4pt} \\
\noindent Faculty Advisor \hfill Date
\\[2\baselineskip]
\noindent \rule{8cm}{0.4pt} \hfill {\rule{4cm}{0.4pt} \\
\noindent Student Activities and Organizations Office \hfill Date
\\[2\baselineskip]

\noindent Approval by the IEEE-Eta Kappa Nu Board of Governors and Executive Director:
\\[2\baselineskip]
\noindent \rule{8cm}{0.4pt} \hfill {\rule{4cm}{0.4pt} \\
\noindent President \hfill Date
\\[2\baselineskip]
\noindent \rule{8cm}{0.4pt} \hfill {\rule{4cm}{0.4pt} \\
\noindent Vice President \hfill Date
\\[2\baselineskip]
\noindent \rule{8cm}{0.4pt}  \hfill {\rule{4cm}{0.4pt} \\
\noindent Executive Director \hfill Date \\

\fancyhead{}

\end{document}